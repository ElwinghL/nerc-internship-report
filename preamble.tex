\usepackage[pdftex]{graphicx}
\usepackage{amsmath}
\usepackage{amssymb}
\usepackage[version-1-compatibility,load=prefixed]{siunitx}
\usepackage{subcaption}
\usepackage{bm}
\usepackage{minted}
\usepackage{datetime}
\usepackage{wrapfig}
\usepackage{tabularx}

\graphicspath{{figures/}{images/}}
\DeclareGraphicsExtensions{.pdf,.eps,.png,.jpg}

\usepackage{hyperref}

\hypersetup{
    colorlinks = true,
    citecolor = purple,
    linkcolor = purple,
    urlcolor = blue,
    pdftex,
    pdfauthor = { Drew Silcock }
    pdftitle = { NERC Internship: Aerial Photogrammetry for the Earth
                 Sciences}
                 pdfsubject = { Earth Sciences }
}

% New definition of square root:
% it renames \sqrt as \oldsqrt
\let\oldsqrt\sqrt
% it defines the new \sqrt in terms of the old one
\def\sqrt{\mathpalette\DHLhksqrt}
\def\DHLhksqrt#1#2{%
\setbox0=\hbox{$#1\oldsqrt{#2\,}$}\dimen0=\ht0
\advance\dimen0-0.2\ht0
\setbox2=\hbox{\vrule height\ht0 depth -\dimen0}%
{\box0\lower0.4pt\box2}}

% For easier integral writing / spacing
\newcommand{\dd}{\; \mathrm{d}}

% For shorter footnote urls
\newcommand{\footurl}[1]{\footnote{\url{#1}}}

% For (number plusminus error) /unit
\newcommand{\sipm}[3]{(#1 $\pm$ #2)\si{#3}}
