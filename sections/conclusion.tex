\section{Conclusion}

In conclusion, aerial photography can provide an accurate 3-dimensional
photogrammetric reconstruction of the topology of the ground, and a useful
corresponding DEM, with an error on the scale of a few centimetres. This is
sufficiently accurate to be able to produce useful DEM for geological purposes.
Importantly, this accuracy can only be achieved using both geotagged photos and
Ground Control Points for the georeferencing. Without the GCPs, the models
exhibit systematic error such as the tilts discussed in Sections
\ref{sec:results/long-ashton} and \ref{sec:results/avon-gorge}.
