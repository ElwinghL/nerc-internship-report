\section{Methods}

This is where I'll explain the methods used.

\subsection{Taking Photos} % Must be a more technical name for this

\begin{itemize}

\item All zenithal, no oblique; see Section
    \ref{subsec:results/zenithal-versus-oblique}

\item CHDK script to take photos every 5 seconds

\item CHDK script to allow Mission Planner to take photos and store GPS data in
log

\end{itemize}

\subsection{Geotagging}

Two main methods:

\subsubsection{Time Offset Method}

This method basically involves telling Mission Planner what the offset between
the first photo taken and the start of the log recording is. It can guess it for
you, but you also need to know the difference between the GPS log time and your
camera time. I should talk about the possible sources of error with this method.

\subsubsection{CAM Dataflash Log Messages}

If you plug your camera into your APM board, you can tell Mission Planner to
take photos for you. It then stores the GPS data (latitude, longitude, altitude,
yaw, pitch, roll and time) at the exact moment it told the camera to take the
photo as a CAM message in the form:

\begin{minted}{python}
CAM, GPSTime, GPSWeek, Lat, Lng, Alt, Roll, Pitch, Yaw
\end{minted}

Mission Planner can then be used to geotag the photos using the dataflash log
containing these CAM messages. This is of course much more accurate than the
time offset method.

One thing you have to take into account is the camera shutter lag, which I
calculated\footnote{Using this website:
\url{http://edwardns.com/shutterlag.html}} to be \SI{90.3 \pm
32.0}{ms} excluding the autofocus lag and \SI{341.3 \pm
140.3}{ms} including it.

\subsection{Ground Control Points}

\begin{itemize}

\item Need at least 10 to 15 GCPs before they have their desired effect

\item Distinguishable crosses are used so as to allow you to pinpoint the exact
centre of the cross.

\item Surveying in the points - equipment used

\end{itemize}

\subsection{Accuracy and Precision Measurements}
% Should this be in methods, results or a separate theory section?

\subsubsection{Calculating the meters per pixel}

Todo

\subsubsection{Ensuring sufficient photo overlap}

Todo
